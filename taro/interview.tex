\direct{This interview by singer-songwriter and actor, Ogaki Tomoya, on the occasion of a performance for the Tenkawa Benzai shrine, was published in the July issue of Entame News, 2017.}

\begin{drama}

\par
\par

\speaker{Ogaki} You've performed many times at temples and shrines. Is the feeling different than at other times?

\par
\par

\speaker{Matsumoto} With the band, Shanion, we had before, we used to give a dedicatory performance every month on the full moon at Shin-yakushiji temple. We got the chance to do this for 6 years, playing more than 50 times there. Even now, I get the opportunity to perform every year at Mizumuro Jinja. Playing before the gods, or for someone who has died and is no longer physically present, you strive to make the mind clear, yes.

\par
\par

Playing before people, you naturally tend to think of technique and how you can give a better performance. But the more I study the classics, the more I felt music has a dedicatory or devotional side to it. I think music is a communication tool connecting this world with the other world. This means a musical offering should not be ensnared by the lures of technique. A quiet mind, susceptible to divine influences, is necessary, don't you think?

\par
\par

A balance between the ancestors and the other world and this world is what I am trying to achieve in my music. This feeling is something I think I realized because of the opportunity I have been given to pursue my career in Nara.

\par
\par

\speaker{Ogaki} This time, it is through Hirai Kay, active nationally in drama that this dedicatory live event at Tenkawa Dai-Benzai Taisha is being realized. Because it is a special unit put together by Hirai just for this event and it will be produced by him, it is being eagerly anticipated.

\par
\par

Wouldn't it be great if I sang my original song, "Hana to Nare." What about you?

\par
\par

\speaker{Matsumoto} I really like Yoshino. And the idea of being able to play at Tenkawa Shrine really makes me happy. I wonder what kind of production Hirai will put together. Slow tempo pieces where all the sound qualities of the shakuhachi can be featured and vigorous pieces, both are fun. Thinking about comping well for a group, I believe I want to contribute a workman-like performance, here 1 millimeter before, there 2 millimeter after. Studying shakuhachi honkyoku, you completely understand making sounds fit in. Finding the right place to do that is important.

\end{drama}

