Taro Matsumoto

Born in Osaka city in 1973.
He was brought up in Nara prefecture, and spent four years studying in high school and university in Australia.
Whilst studying art history and modern thought, he accidentally came upon the the shakuhachi music of the “Watazumido school”, and decided to become a professional shakuhachi player. The music of Watazumido is based on the classic pieces composed by anonymous Zen monks of medieval times. The creator of Watazumido revitalized this classic music with superior technique and the precise logic of breath training. These compositions adopt the representative components of  Japanese spirituality, such as zen, Buddhist chant, native Japanese music and the sounds heard in the natural environment. Matsumoto has studied under Riley Lee, Toshimitsu Ishikawa, and Reishou Yonemura. 

Following are the forms of music Matsumoto has been playing. 

Watazumido honkyoku   classic pieces derived from the 12th century

Kinko ryu honkyoku      classic pieces of the Kinko school

San Kyoku               ensembles with koto and shamisen

He supplies the music for plays and operas,and also is an excellent improviser.


                                         end        
