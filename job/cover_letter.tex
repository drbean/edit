Dear Sir/Madam, 

I found your ad on http://www.tealit.com. I understand you are looking for a native-speaker teacher.

For the last four years ago, I have been working at National United University. But next semester, NUU will be replacing me with an assistant professor. So I am looking for another job. I wish to apply for a position at your school. I think I would be an asset to your English program.

For the first ten years I was in Taiwan at Chinmin Institute of Technology, I was working with lower-level, unmotivated students. For these students, I developed a number of innovative teaching activities. One involved paired dictation, where partners read a dialog, but at the same time write it down. Another was a multiple choice quiz in the form of a basketball/baseball relay. Members of a team stationed around the room mark a ball with the answers to questions posted on the wall, before throwing it to the team member at the front of the room, who competes to throw the ball into the right basket placed on top of the blackboard before it fills up.

Whether I was successful in a lot of my work at Chinmin is debatable. but overall I think I did an excellent job of creating favorable attitudes to English and to school work among these students. I certainly seem to have gained their respect. Some even have jobs where English plays an important role in the work they do.

At NUU, the students have been served better by the education system. They are more able learners. Still, they have been trained to be passive. Getting them to use English in small groups is a challenge. I have continued to use dictation, putting it on the web, at http://web.nuu.edu.tw/~greg/DictationExercises.html. The students listen to stories from http://www.storycorps.org, or other Internet sources, and re-create the text by filling in the blanks for homework. Using a question-answering web application I am now developing, they also write questions about the stories and choose answers, and the application tells them if their question is grammatical, and if the answer is correct.

As you can see, I have been developing my own curriculum. I don't use textbooks in class (although I am willing to use one if it is required.) Textbooks have acted as a starting point from which the curriculum has developed. For example, the topics and much of the material of my Business English classes comes from Longman's Market Leader.
 
In class, where I follow a Cooperative Learning approach, they will do mini-Jigsaw Listening activities, although only a few groups, even in AFLD classes, will use primarily English, despite the activity being quite popular. Another thing I do in class is a popular whole-class fill-in-the-blank exercise projected on the board, where groups are assigned the blanks.

Less popular is a competition between two groups forming one table. One champion from each group asks questions previously written on the board. The two groups of 3,4 members each at the table then vote on which champion won, this then determining the classwork score of the two groups (3 for the winner's group, 2 for the loser's group.) Collusion appears to play a major part in the determining of the winners. This activity, however, is very popular with my AFLD speech classes, debating some topic in groups, where they WILL use only English.

For the 4 exams, which are of only minor importance for their grades compared to homework and classwork, groups will do similar Jigsaw Listening exercises, conducted by 2 sets of student examiners, who will record Chinese use, which lowers their score. At the same time as the Jigsaw activity, students will be matched with an opponent in a Tennis match, where they ask a question and answer their opponent's question. Grammar errors are 'Faults', for which they are allowed a second chance. Incorrectly answered questions are 'Unreturned' and correctly answered questions are 'Returned.' The player with the most points is the winner. I am hoping to re-create this in-class Tennis activity on the web using the question-answering application mentioned above. In class, I am the umpire, deciding the status of the question. 

The problems of most of my students in Taiwan has been a lack of confidence, stress and other forms of discomfort, and a lack of adventurousness. These were the result of unenjoyable experiences in school learning English, the lack of contact with users of the language, and disappointment with their experiments with the language. It would be my aim to give students enjoyable experiences with English, to allow them to develop a personal relationship with a native speaker and give them wings (or encourage them to fly) with the language, by valuing risk-taking with it. 

Leaving NUU now, I am not retiring and I am not slowing down. With the next step in my career, I want to change the Taiwan English-language situation, developing my teaching repertoire and helping my students grow as learners of English. I am confident the future is bright. If your school is also of such a view, I would enjoy working with you, developing confident, relaxed, adventurous and superior speakers of the English language.

Yours sincerely,


Greg Matheson,
Contract Lecturer,
Language Center
National United University
1, Lienda, Miaoli 36003, Taiwan, R.O.C
(E-mail: drbean@freeshell.org)
