\documentclass[a4paper]{article}
\usepackage[T1]{fontenc}
% \usepackage[absolute,noshowtext,showboxes]{textpos}
\usepackage[absolute,showboxes]{textpos}
% \textblockorigin{0.10cm}{1.00cm}
\textblockorigin{0.00cm}{0.00cm} %HPLaserJet5000LE
\usepackage{texdraw}
\pagestyle{empty}
\setlength{\unitlength}{1cm}

\newcommand{\myIdentifier}[0]{
graham
}

\newcommand{\myAcontent}[0]{
It's important to have clear goals, running a business or writing software. It is very difficult to plan, because you can't see the future, but you need to be very clear about your goals. If you are very clear about WHAT you want to do, it is easy to decide HOW to do it.
Before talking to users, you need to do a lot of planning. But it is also very important to talk to the users. You need to see what they think. Their feedback will help you plan.

}

\newcommand{\myBcontent}[0]{
Writing down business plans is hard work, but if you think about it a lot, it will help you be clear about what your business is doing. The plan you have written down can also be shown to many different people who want to see your business plan. The hard work writing down the business plan should not be regarded as a waste of time. It helps you decide what your goals are.

}

\newcommand{\myCcontent}[0]{
The fastest-growing businesses get all the people in the business excited about their plans. Everyone feels the company's goals are their own goals. But you can't give people a plan and expect them to get excited about it.  People get excited about the boss's goals, and their role, only when they feel they are involved in creating them and finding new life goals for themselves, This doesn't happen when people feel they are just following the boss's goals.

}

\newcommand{\myDcontent}[0]{
William Kendall became very good at planning ahead. He couldn't sell his previous company, because people thought they wouldn't run it as well as him. So, starting his next company, he immediately started thinking how to sell it. He went to a very big company, and asked them if they wanted to invest ten percent. They did. And now the big company has bought all of his company. That was his plan.

}

\newcommand{\mycard}[5]{%
	\vspace{0.1cm}
	\small #1 #2
	\par
	\parbox[t][6.7cm][c]{9.5cm}{%
	\hspace{0.1cm} \Large#3\\
	\normalsize#4 #5
	}
}

\begin{document}
\fontfamily{hlst}\fontseries{b}\fontshape{n}\selectfont
% \begin{picture}(15,20)(+4.8,-22.05)
% \begin{tabular}[t]{*{2}{|p{10.05cm}}|}

\begin{textblock}{8}(0,0)
\textblocklabel{picture1}
\mycard{1/1}{\myIdentifier}{\parbox{9.0cm}{A:
\myAcontent
}}{}{} 
\end{textblock}

\begin{textblock}{8}(8,0)
\textblocklabel{picture2}
\mycard{1/1}{\myIdentifier}{\parbox{9.0cm}{B:
\myBcontent
}}{}{} 
\end{textblock}

\begin{textblock}{8}(0,4)
\textblocklabel{picture3}
\mycard{1/1}{\myIdentifier}{\parbox{9.0cm}{C:
\myCcontent
}}{}{} 
\end{textblock}

\begin{textblock}{8}(8,4)
\textblocklabel{picture4}
\mycard{1/1}{\myIdentifier}{\parbox{9.0cm}{D:
\myDcontent
}}{}{} 
\end{textblock}

\begin{textblock}{8}(0,8)
\textblocklabel{picture5}
\mycard{1/1}{\myIdentifier}{\parbox{9.0cm}{A:
\myAcontent
}}{}{} 
\end{textblock}

\begin{textblock}{8}(8,8)
\textblocklabel{picture6}
\mycard{1/1}{\myIdentifier}{\parbox{9.0cm}{B:
\myBcontent
}}{}{} 
\end{textblock}

\begin{textblock}{8}(0,12)
\textblocklabel{picture7}
\mycard{1/1}{\myIdentifier}{\parbox{9.0cm}{C:
\myCcontent
}}{}{} 
\end{textblock}

\begin{textblock}{8}(8,12)
\textblocklabel{picture8}
\mycard{1/1}{\myIdentifier}{\parbox{9.0cm}{D:
\myDcontent
}}{}{} 
\end{textblock}

\end{document}


